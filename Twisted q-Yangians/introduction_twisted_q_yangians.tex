\section{Introduction}
    \subsection{Notations}
        Let $\sfX$ be an affine Dynkin diagram - untwisted or twisted - of a classical type (see \cite[Chapter 4, Tables Aff 2 and 3, p. 55]{kac_infinite_dimensional_lie_algebras}), let $\g(\sfX)$ be the associated (affine) Kac-Moody algebra, let $\g_N \subset \sl_N$ be the underlying finite-dimensional simple Lie algebra, and let $\bar{\sfX}$ be the Dynkin diagram associated to $\g_N$. Let $\sigma$ be a finite-order automorphism of $\bar{\sfX}$. This gives rise to a Lie algebra automorphism of $\g_N$, which in turn gives rise to an automorphism of the loop algebra $\Loop\g_N \cong \g_N \tensor_{\bbC} \bbC[t^{\pm 1}]$. Then, write $\Loop^{\sigma}\g_N$ for the fixed-point subalgebra; this is the Lie algebra that we will be mainly working with. 

    \subsection{Overview}
        Our goal here is to generalise the results of \cite{molev_ragoucy_sorba_twisted_q_yangians_type_A}. Therein, the authors constructed the so-called \say{twisted $q$-Yangians}, which are coideal subalgebras of the quantum loop algebras $\calU_q(\Loop\sl_N)$ obtained by replacing the rational solutions to the QYBE\footnote{Quantum Yang-Baxter equation.} in the twisted Yangian construction of Olshanskii (cf. \cite{olshanski_twisted_yangians_and_infinite_dimensional_classical_lie_algebras} or \cite[Chapter 2]{molev_yangians_and_classical_lie_algebras}) with trigonometric ones. Doing so gives rise to trigonometric-type solutions to the bQYBE\footnote{Boundary quantum Yang-Baxter equation.} (cf. definition \ref{def: boundary_quantum_yang_baxter_equations}), thereby obtaining new coideal subalgebras of $\calU_q(\Loop\sl_N)$ different from the usual quantum symmetric pairs. We would now like to generalise this construction to quantum loop algebras of all remaining non-exceptional affine types (in the Kac classification, which can be found e.g. in \cite[Chapter 4]{kac_infinite_dimensional_lie_algebras}), and in particular to the twisted types:
            $$\sfA_2^{(2)}$$
            $$\sfA_{2\ell}^{(2)}, \ell \geq 2$$
            $$\sfA_{2\ell - 1}^{(2)}, \sfA_{2\ell - 1}^{(2')}, \ell \geq 3$$
            $$\sfD_{\ell}^{(2)}, \ell \geq 3$$
            $$\sfD_4^{(3)}$$
        and we write $\sfA_{2\ell - 1}^{(2')}$ to mean the second realisation of the twisted affine type $\sfA_{2\ell - 1}$, wherein the affinising vertex of the Dynkin diagram is the long simple root (see \cite[Chapter 4, Tables Aff 2 and 3, p. 55]{kac_infinite_dimensional_lie_algebras}). As for the untwisted affine types, they have been touched upon in \cite{regelskis_vlaar_reflection_matrices_coideal_subalgebras}, wherein the authors established twisted $q$-Yangians as coideal subalgebras of the quantum loop algebras of said untwisted types.  
        
        As a prerequisite, it is necessary that we know how to realise the quantum loop algebras of these types in an R-matrix presentation, so that afterwards, the corresponding R-matrices can be used for the construction of twisted $q$-Yangians inside these quantum loop algebras. Fortunately, a forthcoming work by Guay-Regelskis-Wendlandt will supply us with such R-matrix presentations. Therefore, we will be giving an overview of their results, and then move on to describing twisted $q$-Yangians.

    \subsection{\texorpdfstring{Defining twisted $q$-Yangians}{}}
        Let $\g_N \subset \sl_N$ be a finite-dimensional simple Lie algebra of a classical type in the Cartan-Killing classification.
    
        To begin, unlike in \cite{molev_ragoucy_sorba_twisted_q_yangians_type_A} where the authors defined twisted $q$-Yangians embeddedly as associative subalgebras of $\calU_q(\Loop\sl_N)$ first of all, let us follow \cite{regelskis_vlaar_reflection_matrices_coideal_subalgebras} and define twisted $q$-Yangians as reflection algebras in the sense of \cite{guay_regelskis_twisted_yangians_for_symmetric_pairs_of_types_BCD}, but with the rational matrices therein replaced by trigonometric ones. We will then show that these reflection-type algebras indeed embed as coideal subalgebras into $\calU_q(\Loop^{\sigma}\g_N)$, and then compute their centres by establishing a $q$-analogue of Sklyanin determinants.

        Next, we will see if the constructed twisted $q$-Yangians admit evaluation homomorphisms to the finite-type quantum groups $\calU_q(\g_N)$. Said evaluation homomorphisms are to be compatible with the specialisation $q \to 1$ as well, which is to say that after specialisation, one ought to recover the usual evaluation homomorphisms of the usual twisted Yangians.