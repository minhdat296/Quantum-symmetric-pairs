\section{\texorpdfstring{Reflection algebras of types $\BCDzero$}{}}
    \subsection{Reflection algebras of type \texorpdfstring{$\sfB, \sfC, \sfD$}{}}
        The following notions are due to Guay-Regelskis (see \cite{guay_regelskis_twisted_yangians_for_symmetric_pairs_of_types_BCD}).
        \begin{definition}[(Extended) reflection algebras] \label{def: (extended)_reflection_algebras} 
            \begin{enumerate}
                \item 
                \item 
                \item 
            \end{enumerate}
        \end{definition}
        \begin{convention}[IMO reflection algebras] \label{conv: IMO_reflection_algebras}
            In \cite{isaev_molev_ogievetsky_fusion_for_brauer_algebras_2}, a notion of reflection algebra was also given. In the present document, it shall be denoted by:
                $$\UXB(\g_N, \id)$$
            as it is obtained as the associative algebra generated by the coefficients of the matrix entries of $S(u)$, subjected to the $\BCDzero$ reflection equation and the unitary condition, but \textit{without} the symmetry relation (cf. \cite[Definition 3.1]{isaev_molev_ogievetsky_fusion_for_brauer_algebras_2}). Also, as proposition \ref{prop: mapping_IMO_reflection_algebras_to_extended_yangians} will point out, its generating matrix - which shall also be denoted by $B(u)$ - admits a somewhat different image when mapped to the extended Yangian $\calX(\g_N)$ from that of the Guay-Regelskis $\BCDzero$ unitary reflection algebra. So that we do not confuse it with the various kinds of reflection algebras (of type $\BCDzero$) in the sense of Guay-Regelskis that were mentioned above (particularly, with the unitary reflection algebra $\UB(\g_N, \id)$), we shall refer to $\UXB(\g_N, \id)$ as the \textbf{Isaev-Molev-Ogievetsky (IMO) reflection algebra}.
        \end{convention}
        \begin{remark}
            The IMO reflection algebra is not quite the same as the reflection algebra of Molev-Ragoucy from \cite{molev_ragoucy_representations_of_reflection_algebras}, which should correspond to the twisted Yangian of type $\sfA \romanthree$.
        \end{remark}
        From now on, let $\g_N$ be of one of the types $\sfB, \sfC, \sfD$ in the Cartan-Killing Classification, and let $\vartheta := \id$.

        \begin{proposition}[A homomorphism $\UXB(\g_N, \id) \to \calX(\g_N)$] \label{prop: mapping_IMO_reflection_algebras_to_extended_yangians}
            There is an algebra homomorphism:
                \begin{equation} \label{equation: mapping_IMO_reflection_algebras_to_extended_yangians}
                    \Phi: \UXB(\g_N, \id)[\![u^{-1}]\!] \to \calX(\g_N)[\![u^{-1}]\!]
                \end{equation}
            determined by:
                $$\Phi(B(u)) := T\left(u - \frac12 \kappa\right) T\left(-u + \frac12 \kappa\right)^{-1}$$
            thus defining an algebra homomorphism $\UXB(\g_N, \id) \to \calX(\g_N)$.
        \end{proposition}
            \begin{proof}
                See \cite[Proposition 3.2]{isaev_molev_ogievetsky_fusion_for_brauer_algebras_2}
            \end{proof}
        \begin{remark}
            Later on, we will see that the homomorphism \eqref{equation: mapping_IMO_reflection_algebras_to_extended_yangians} is neither injective nor surjective in general. See theorem \ref{theorem: IMO_reflection_algebras_vs_BCD0_twisted_yangians} and corollary \ref{coro: IMO_reflection_algebras_vs_BCD0_twisted_yangians} for more details.
        \end{remark}

    \subsection{IMO reflection algebras vs. twisted Yangians of type \texorpdfstring{$\BCDzero$}{}}
        In this subsection, we work only with the $\BCDzero$ symmetric pair:
            $$(\g_N, \vartheta) = (\g_N, \id)$$

        The central question that we would like to spend the subsection to answer is as follows.
        \begin{question}
            Does the algebra homomorphism \eqref{equation: mapping_IMO_reflection_algebras_to_extended_yangians} factor in the following manner:
                \begin{equation} \label{diagram: mapping_IMO_reflection_algebras_to_extended_twisted_yangians}
                    \begin{tikzcd}
                        {\UXB(\g_N, \id)[\![u^{-1}]\!]} & {\calX^{\tw}(\g_N, \id)[\![u^{-1}]\!]} \\
                        & {\calX(\g_N)[\![u^{-1}]\!]}
                        \arrow["{\exists ? \Phi^{\tw}}", dashed, from=1-1, to=1-2]
                        \arrow["\Phi"', from=1-1, to=2-2]
                        \arrow[hook, from=1-2, to=2-2]
                    \end{tikzcd}
                \end{equation}
            wherein the unlabelled arrow is the canonical algebra embedding; in other words, do matrix entries of:
                $$\Phi(B(u)) := T\left(u - \frac12 \kappa\right) T\left(-u + \frac12 \kappa\right)^{-1}$$
            actually lie inside the subalgebra $\calX^{\tw}(\g_N, \id)[\![u^{-1}]\!] \subset \calX(\g_N)[\![u^{-1}]\!]$ ?
        \end{question}
    
        Homomorphisms map generators to generators, so let us first of all consider if and how the expression for $\Phi( B(u) )$ as above differs from that for the transfer matrix $S(u)$, whose entries generate $\calX^{\tw}(\g_N, \id)$. To this end, recall from definition \ref{def: (extended)_twisted_yangians} that:
            $$S(u) = T\left(u - \frac12 \kappa\right) T\left(-u + \frac12 \kappa\right)^t$$
        Indeed, this is different from the expression for $\Phi(B(u))$ from proposition \ref{prop: mapping_IMO_reflection_algebras_to_extended_yangians}, so one way for an algebra homomorphism $\Phi^{\tw}$ as in diagram \eqref{diagram: mapping_IMO_reflection_algebras_to_extended_twisted_yangians} to exist is for there to be an algebra automorphism $\beta^{\tw} \in \Aut_{\Assoc\Alg}( \calX^{\tw}(\g_N, \id)[\![u^{-1}]\!] )$ such that the equation $(\beta^{\tw} \circ \Phi)( B(u) ) = S(u)$ holds in $\Mat_N(\calX^{\tw}(\g_N, \id)[\![u^{-1}]\!])$. For this to hold, though, one must already have that $\im \Phi \subset \calX^{\tw}( \g_N, \id )[\![u^{-1}]\!]$, which is the very thing we are trying to verify, so instead, let us instead search for an algebra automorphism:
            $$\beta \in \Aut_{\Assoc\Alg}( \calX(\g_N)[\![u^{-1}]\!] )$$
        such that:
            \begin{equation} \label{equation: B_matrix_S_matrix_compatibility}
                (\beta \circ \Phi)( B(u) ) = S(u)
            \end{equation}
        and then afterwards, we would take $\beta^{\tw} = \beta|_{ \calX^{\tw}( \g_N, \id )[\![u^{-1}]\!] }$ to get an automorphism of the subalgebra $\calX^{\tw}(\g_N, \id)[\![u^{-1}]\!] \subset \calX(\g_N)[\![u^{-1}]\!]$. The original question can now be rephrased in the following manner.
        \begin{question}
            Does there exist an algebra automorphism $\beta \in \Aut_{\Assoc\Alg}( \calX(\g_N)[\![u^{-1}]\!] )$ fitting into the following commutative diagram ?
                \begin{equation} \label{diagram: mapping_IMO_reflection_algebras_to_extended_twisted_yangians_via_untwisted_automorphisms}
                    \begin{tikzcd}
                    {\UXB(\g_N, \id)[\![u^{-1}]\!]} & {\calX^{\tw}(\g_N, \id)[\![u^{-1}]\!]} \\
                    {\calX(\g_N)[\![u^{-1}]\!]} & {\calX(\g_N)[\![u^{-1}]\!]}
                    \arrow["{\exists ? \Phi^{\tw}}", dashed, from=1-1, to=1-2]
                    \arrow["\Phi"', from=1-1, to=2-1]
                    \arrow[hook, from=1-2, to=2-2]
                    \arrow["\beta", from=2-1, to=2-2]
                    \end{tikzcd}
                \end{equation}
        \end{question}
        
        A natural case to consider is when $\beta$ is the inner automorphism defined on the transfer matrix $T(u) \in \Mat_N( \calX(\g_N)[\![u^{-1}]\!] )$ by:
            $$\beta( T(u) ) := b(u) \cdot T(u)$$
        for some invertible central scalar series:
            $$b(u) \in \calZ(\g_N)[\![u^{-1}]\!]^{\x}$$
        In this case, we have:
            $$
                \begin{aligned}
                    (\beta \circ \Phi)( B(u) ) & = \beta\left( T\left(u - \frac12 \kappa\right) T\left(-u + \frac12 \kappa\right)^{-1} \right)
                    \\
                    & = \beta\left( T\left(u - \frac12 \kappa\right) \right) \beta\left( T\left(-u + \frac12 \kappa\right) \right)^{-1}
                    \\
                    & = b\left(u - \frac12 \kappa\right) T\left(u - \frac12 \kappa\right) \cdot T\left(-u + \frac12 \kappa\right)^{-1} b\left(-u + \frac12 \kappa\right)^{-1}
                    \\
                    & = b\left(u - \frac12 \kappa\right) b\left(-u + \frac12 \kappa\right)^{-1} \cdot T\left(u - \frac12 \kappa\right) T\left(-u + \frac12 \kappa\right)^{-1}
                    \\
                    & = b\left(u - \frac12 \kappa\right) b\left(-u + \frac12 \kappa\right)^{-1} \cdot \Phi( B(u) )
                \end{aligned}
            $$
        For brevity, let:
            $$b^{\tw}(u) := b\left(u - \frac12 \kappa\right) b\left(-u + \frac12 \kappa\right)^{-1}$$
        and then equation \eqref{equation: B_matrix_S_matrix_compatibility} thus reads:
            $$S(u) = (\beta \circ \Phi)( B(u) ) = b^{\tw}(u) \cdot \Phi( B(u) )$$
        We must now check if this is compatible with the unitary condition:
            $$B(u) B(-u) = 1$$
        coming from the definition of $\UXB(\g_N, \id)$, as well as the twisted quantum contraction formula (see lemma \ref{lemma: twisted_quantum_contractions}):
            $$S(u) S(-u) = z^{\tw}(u)$$
        To this end, consider the following:
            $$
                \begin{aligned}
                    z^{\tw}(u) & = S(u) S(-u)
                    \\
                    & = \left( b^{\tw}(u) \cdot \Phi( B(u) ) \right) \cdot \left( b^{\tw}(-u) \cdot \Phi( B(-u) ) \right)
                    \\
                    & = b^{\tw}(u) b^{\tw}(-u) \cdot \Phi( B(u) B(-u) )
                    \\
                    & = b^{\tw}(u) b^{\tw}(-u) \cdot \Phi(1)
                    \\
                    & = b^{\tw}(u) b^{\tw}(-u)
                \end{aligned}
            $$
        This is to hold if the homomorphism $\Phi^{\tw}$ is to exist, and thus we are led to the following question.
        \begin{question} \label{question: square_roots_of_twsited_quantum_contractions}
            Is it possible to factorise:
                \begin{equation} \label{equation: square_roots_of_twisted_quantum_contractions}
                    z^{\tw}(u) = b^{\tw}(u) b^{\tw}(-u)
                \end{equation}
            for some $b^{\tw}(u) \in \calZ(\g_N)[\![u^{-1}]\!]^{\x}$, or better yet, for some $b^{\tw}(u) \in \calZ^{\tw}(\g_N, \id)[\![u^{-1}]\!]^{\x}$ ?
        \end{question}
        \begin{remark}
            This is consistent with the fact that $z^{\tw}(u)$ is an even formal power series \textit{a priori}.
        \end{remark}

        The following was noted already at the end of \cite[Section 2]{guay_regelskis_twisted_yangians_for_symmetric_pairs_of_types_BCD}, but no proof was provided, so we supply one here.
        \begin{lemma} \label{lemma: quantum_pfaffians}
            There exists an element:
                $$y(u) \in \calZ(\g_N)[\![u^{-1}]\!]$$
            so that the quantum contraction $z(u) \in \calZ(\g_N)[\![u^{-1}]\!]^{\x}$ as in lemma \ref{lemma: quantum_contractions} factorises in the following manner:
                \begin{equation} \label{equation: quantum_pfaffians}
                    z(u) = y(u) y(u + \kappa)
                \end{equation}
        \end{lemma}
            \begin{proof}
                First of all, since $z(u) \in \calX(\g_N)[\![u^{-1}]\!]^{\x}$ and since $\calX(\g_N)[\![u^{-1}]\!]^{\x}$ is multiplicatively closed, we can take $y(u) \in \calX(\g_N)[\![u^{-1}]\!]^{\x}$ too. Such a series is of the form:
                    $$y(u) = 1 + \sum_{r \geq 0} y^{(r)} u^{-r - 1}$$
                \textit{a priori}, and thus:
                    $$z(u) = y(u) y(u + \kappa) = \left( 1 + \sum_{m \geq 0} y^{(m)} u^{-m - 1} \right) \left( 1 + \sum_{m \geq 0} y^{(m)} (u + \kappa)^{-m - 1} \right)$$
                Now, we note that:
                    $$\frac{1}{u + \kappa} = \frac{u^{-1}}{1 + \kappa u^{-1}} = u^{-1} \sum_{n \geq 0} (-\kappa u^{-1})^n = \sum_{n \geq 0} (-\kappa)^n u^{-n - 1}$$
                This then allows us to write:
                    $$
                        \begin{aligned}
                            z(u) & = \left( 1 + \sum_{m \geq 0} y^{(m)} u^{-m - 1} \right) \left( 1 + \sum_{m \geq 0} y^{(m)} (u + \kappa)^{-m - 1} \right)
                            \\
                            & = \left( 1 + \sum_{m \geq 0} y^{(m)} u^{-m - 1} \right) \left( 1 + \sum_{m \geq 0} y^{(m)} \left( \sum_{n \geq 0} (-\kappa)^n u^{-n - 1} \right)^{-m - 1} \right)
                            \\
                            & =
                            \begin{aligned}
                                & 1 + \sum_{m \geq 0} y^{(m)} u^{-m - 1} + \sum_{m \geq 0} y^{(m)} \left( \sum_{n \geq 0} (-\kappa)^n u^{-n - 1} \right)^{-m - 1}
                                \\
                                & \quad + \left( \sum_{l \geq 0} y^{(l)} u^{-l - 1} \right)\left( \sum_{m \geq 0} y^{(m)} \left( \sum_{n \geq 0} (-\kappa)^n u^{-n - 1} \right)^{-m - 1} \right)
                            \end{aligned}
                            \\
                            & =
                            \begin{aligned}
                                & (...)
                                \\
                                & \quad + \sum_{l \geq 0} \sum_{m \geq 0} y^{(l)} u^{-l - 1} \cdot y^{(m)} \left( \sum_{n \geq 0} (-\kappa)^n u^{-n - 1} \right)^{-m - 1}
                            \end{aligned}
                            \\
                            & =
                            \begin{aligned}
                                & (...)
                                \\
                                & \quad + \sum_{l \geq 0} \sum_{m \geq 0} y^{(l)} y^{(m)} \cdot u^{-l - 1} \cdot \left( \frac{1}{u + \kappa} \right)^{-m - 1}
                            \end{aligned}
                            \\
                            & =
                            \begin{aligned}
                                & (...)
                                \\
                                & \quad + \sum_{l \geq 0} \sum_{m \geq 0} y^{(l)} y^{(m)} \cdot u^{-l - 1} \cdot (u + \kappa)^{m + 1}
                            \end{aligned}
                            \\
                            & = 
                            \begin{aligned}
                                & (...)
                                \\
                                & \quad + \sum_{l \geq 0} \sum_{m \geq 0} y^{(l)} y^{(m)} \cdot u^{-l - 1} \cdot \sum_{k = 0}^{m + 1} \binom{m + 1}{k} \kappa^k u^{m + 1 - k}
                            \end{aligned}
                            \\
                            & =
                            \begin{aligned}
                                & (...)
                                \\
                                & \quad + \sum_{l \geq 0} \sum_{m \geq 0} y^{(l)} y^{(m)} \cdot \sum_{k = 0}^{m + 1} \binom{m + 1}{k} \kappa^k u^{m - l - k}
                            \end{aligned}
                        \end{aligned}
                    $$
                Since $z(u) \in \calZ(\g_N)[\![u^{-1}]\!]^{\x}$, it can be written as:
                    $$z(u) = 1 + \sum_{m \geq 0} z^{(m)} u^{-m - 1}$$
                with $z^{(m)} \in \calZ(\g_N)$, and since these coefficients are algebraically independent from one another and generate the centre $\calZ(\g_N)$. Through homogeneity, we then infer that:
                    $$z^{(0)} = 2y^{(0)}$$
                    $$\vdots$$
                    $$
                        z^{(m)} =
                        \begin{aligned}
                            & (1 + (-\kappa)^m) y^{(m)}
                            \\
                            & \quad + \left( \sum_{l \geq 0} \sum_{m \geq 0} y^{(l)} y^{(m)} \cdot \sum_{k = 0}^{m + 1} \binom{m + 1}{k} \kappa^k u^{m - l - k} \right)^{(m)} 
                        \end{aligned}
                    $$
                wherein $(-)^{(m)}$ means the degree $-m - 1$ coefficient. This determines $y(u)$, so we have existence.

                We also know from lemma \ref{lemma: centres_of_extended_untwisted_yangians} that the coefficients $z^{(m)} \in \calZ(\g_N)$, so by induction, one can show that $y^{(m)} \in \calZ(\g_N)$ necessarily as well, by arguing in the following manner. To this end, recall that $\frac{1}{u + \kappa} = \sum_{n \geq 0} (-\kappa)^n u^{-n - 1}$, which then gives us:
                    $$y(u + \kappa) = 1 + \sum_{m \geq 0} y^{(m)} \left( \sum_{n \geq 0} (-\kappa)^n u^{-n - 1} \right)^{-m - 1}$$
                With this in mind, along with the fact that $z(u) \in \calZ(\g_N)[\![u^{-1}]\!]$, let us then consider the following for all $X(u) \in \calX(\g_N)[\![u^{-1}]\!]$:
                    $$
                        \begin{aligned}
                            0 & = [z(u), X(u)]
                            \\
                            & = [y(u) y(u + \kappa), X(u)]
                            \\
                            & =
                            \begin{aligned}
                                & y(u) \cdot [y(u + \kappa), X(u)]
                                \\
                                & \quad + [y(u), X(u)] \cdot y(u + \kappa)
                            \end{aligned}
                            \\
                            & =
                            \begin{aligned}
                                & \left( 1 + \sum_{l \geq 0} y^{(l)} u^{-l - 1} \right) \cdot \left[ 1 + \sum_{m \geq 0} y^{(m)} \left( \sum_{n \geq 0} (-\kappa)^n u^{-n - 1} \right)^{-m - 1}, X(u) \right]
                                \\
                                & \quad + \left[ 1 + \sum_{l \geq 0} y^{(l)} u^{-l - 1}, X(u) \right] \cdot \left( 1 + \sum_{m \geq 0} y^{(m)} \left( \sum_{n \geq 0} (-\kappa)^n u^{-n - 1} \right)^{-m - 1} \right) 
                            \end{aligned}
                            \\
                            & =
                            \begin{aligned}
                                & \left( 1 + \sum_{l \geq 0} y^{(l)} u^{-l - 1} \right) \cdot \sum_{m \geq 0} \left[ y^{(m)} \left( \sum_{n \geq 0} (-\kappa)^n u^{-n - 1} \right)^{-m - 1}, X(u) \right]
                                \\
                                & \quad + \sum_{l \geq 0} \left[ y^{(l)} u^{-l - 1}, X(u) \right] \cdot \left( 1 + \sum_{m \geq 0} y^{(m)} \left( \sum_{n \geq 0} (-\kappa)^n u^{-n - 1} \right)^{-m - 1} \right) 
                            \end{aligned}
                        \end{aligned}
                    $$
                From this, we see that we must have:
                    $$[y^{(m)}, X(u)] = 0$$
                for all $m \geq 0$ and all $X(u) \in \calX(\g_N)[\![u^{-1}]\!]$, thus proving that:
                    $$y(u) \in \calZ(\g_N)[\![u^{-1}]\!]$$
            \end{proof}
        Once again, the following was already noted in \cite{guay_regelskis_twisted_yangians_for_symmetric_pairs_of_types_BCD}, particular at the beginning of Theorem 3.1, wherein the authors used the notations $w(u) = z^{\tw}(u)$ and $q(u) = y^{\tw}(u)$.
        \begin{lemma} \label{lemma: twisted_quantum_pfaffians}
            The twisted quantum contraction $z^{\tw}(u) \in \calZ^{\tw}(\g_N, \id)[\![u^{-1}]\!]^{\x}$ from lemma \ref{lemma: twisted_quantum_contractions} factorises in the following manner:
                \begin{equation} \label{equation: twisted_quantum_pfaffians}
                    z^{\tw}(u) = y^{\tw}(u) y^{\tw}(u + \kappa)
                \end{equation}
            wherein:
                $$y^{\tw}(u) := y\left( u - \frac12 \kappa \right) y\left( -u + \frac12 \kappa \right)$$
            and $y(u) \in \calZ(\g_N)[\![u^{-1}]\!]^{\x}$ is as in lemma \ref{lemma: quantum_pfaffians}. Moreover, we have:
                $$y^{\tw}(u) \in \calZ^{\tw}(\g_N, \id)[\![u^{-1}]\!]$$
        \end{lemma}
            \begin{proof}
                From lemma \ref{lemma: twisted_quantum_contractions}, we know that $z^{\tw}(u) = z\left( u - \frac12 \kappa \right) z\left( -u + \frac12 \kappa \right)$. By combining this with equation \eqref{equation: quantum_pfaffians}, we then get:
                    $$
                        \begin{aligned}
                            z^{\tw}(u) & = z\left( u - \frac12 \kappa \right) z\left( -u + \frac12 \kappa \right)
                            \\
                            & = y\left( u - \frac12 \kappa \right) y\left( u + \frac12 \kappa \right) \cdot y\left( -u + \frac12 \kappa \right) y\left( -u + \frac32 \kappa \right)
                            \\
                            & = y\left( u - \frac12 \kappa \right) y\left( -u + \frac12 \kappa \right) \cdot y\left( u + \frac12 \kappa \right) y\left( -u + \frac32 \kappa \right)
                        \end{aligned}
                    $$
                Setting $y^{\tw}(u) := y\left( u - \frac12 \kappa \right) y\left( -u + \frac12 \kappa \right)$ then yields us equation \eqref{equation: twisted_quantum_pfaffians}.

                Then, by arguing as in the proof of lemma \ref{lemma: quantum_pfaffians}, one sees furthermore that $y^{\tw}(u) \in \calZ^{\tw}(\g_N, \id)[\![u^{-1}]\!]$.
            \end{proof}
        \begin{remark}
            We can rewrite equation \eqref{equation: twisted_quantum_pfaffians} into the following form, which has the advantage of being more symmetric and more clearly consistent with the evenness of $z^{\tw}(u)$:
                \begin{equation} \label{equation: symmetrised_twisted_quantum_pfaffians}
                    z^{\tw}(u) = y^{\tw}(u) y^{\tw}(-u)
                \end{equation}
        \end{remark}
        We can now answer question \ref{question: square_roots_of_twsited_quantum_contractions} in the affirmative by taking:
            $$b^{\tw}(u) := y^{\tw}(u)$$
        Let us now compute $b(u) \in \calZ(\g_N)[\![u^{-1}]\!]^{\x}$ in terms of $y(u) \in \calZ(\g_N)[\![u^{-1}]\!]^{\x}$; as these are invertible series, they can be written in the form:
            $$b(u) := 1 + \sum_{m \geq 0} b^{(m)} u^{-m - 1}$$
            $$y(u) := 1 + \sum_{m \geq 0} y^{(m)} u^{-m - 1}$$
        for some $b^{(m)}, y^{(m)} \in \calZ(\g_N)$. By convention, we have $b^{\tw}(u) := b\left( u - \frac12 \kappa \right) b\left( -u + \frac12 \kappa \right)^{-1}$ and $y^{\tw}(u) := y\left( u - \frac12 \kappa \right) y\left( -u + \frac12 \kappa \right)$, which are respectively equivalent to:
            $$b^{\tw}\left( -u - \frac12 \kappa \right) = b(-u) b(u)^{-1}$$
            $$y^{\tw}\left( -u - \frac12 \kappa \right) = y(-u) y(u)$$
        Next, let us compute $b(u)^{-1}$; to this end, set:
            $$b(u)^{-1} := 1 + \sum_{n \geq 0} c^{(n)} u^{-n - 1}$$
        for some $c^{(n)} \in \calZ(\g_N)$ (again, the $0^{th}$ degree term is $1$ as the series is invertible) and then consider the following:
            $$
                \begin{aligned}
                    1 & = b(u) b(u)^{-1}
                    \\
                    & = \left( 1 + \sum_{m \geq 0} b^{(m)} u^{-m - 1} \right) \left( 1 + \sum_{n \geq 0} c^{(n)} u^{-n - 1} \right)
                    \\
                    & = 1 + \sum_{m \geq 0} b^{(m)} u^{-m - 1} + \sum_{n \geq 0} c^{(n)} u^{-m - 1} + \left( \sum_{m \geq 0} b^{(m)} u^{-m - 1} \right) \left( \sum_{n \geq 0} c^{(n)} u^{-n - 1} \right)
                    \\
                    & = 1 + \sum_{m \geq 0} ( b^{(m)} + c^{(m)} ) u^{-m - 1} + \sum_{m \geq 0} \sum_{n \geq 0} b^{(m)} c^{(n)} u^{-m - n - 2}
                    \\
                    & = 1 + \sum_{l \geq 0} \left( b^{(l)} + c^{(l)} + \sum_{ \substack{m, n \geq 0 \\ m + n = l} } b^{(m)} c^{(n)} \right) u^{-l - 1}
                \end{aligned}
            $$
        By homogeneity, we then have for all $l \geq 0$ that $b^{(l)} + c^{(l)} + \sum_{ \substack{m, n \geq 0 \\ m + n = l} } b^{(m)} c^{(n)} = 0$, or equivalently, the following recursive formula for the coefficients $c^{(l)}$ of $b(u)^{-1}$:
            $$c^{(l)} = -b^{(l)} - \sum_{m = 0}^l b^{(m)} c^{(l - m)}$$
        This then allows us to compute:
            $$
                \begin{aligned}
                    b(-u) b(u)^{-1} & = \left( 1 + \sum_{m \geq 0} b^{(m)} (-u)^{-m - 1} \right) \left( 1 + \sum_{n \geq 0} c^{(n)} u^{-n - 1} \right)
                    \\
                    & = \left( 1 + \sum_{m \geq 0} b^{(m)} (-u)^{-m - 1} \right) \left( 1 + \sum_{n \geq 0} c^{(n)} u^{-n - 1} \right)
                    \\
                    & = 1 + \sum_{m \geq 0} b^{(m)} (-u)^{-m - 1} + \sum_{n \geq 0} c^{(n)} u^{-n - 1} + \left( \sum_{m \geq 0} b^{(m)} (-u)^{-m - 1} \right) \left( \sum_{n \geq 0} c^{(n)} u^{-n - 1} \right)
                    \\
                    & = 1 + \sum_{m \geq 0} ( (-1)^{-m - 1} b^{(m)} + c^{(n)} ) u^{-m - 1} + \sum_{m \geq 0} \sum_{n \geq 0} (-1)^{-m - 1} b^{(m)} c^{(n)} u^{-m - n - 2}
                    \\
                    & = 1 + \sum_{l \geq 0} \left( (-1)^{-l - 1} b^{(l)} + c^{(l)} + \sum_{ \substack{m, n \geq 0\\m + n = l} } (-1)^{-m - 1} b^{(m)} c^{(n)} \right) u^{-l - 1}
                    \\
                    & = 1 + \sum_{l \geq 0} \left( (-1)^{-l - 1} b^{(l)} + \left( -b^{(l)} - \sum_{m = 0}^l b^{(m)} c^{(l - m)} \right) + \sum_{m = 0}^l (-1)^{-m - 1} b^{(m)} c^{(l - m)} \right) u^{-l - 1}
                    \\
                    & = 1 + \sum_{l \geq 0} \left( 2b^{(2l + 1)} - \sum_{ \substack{ 0 \leq m \leq l \\ \text{$m$ even} } } b^{(m)} c^{(l - m)} \right) u^{-l - 1}
                \end{aligned}
            $$
        At the same time, we have:
            $$
                \begin{aligned}
                    y(-u) y(u) & = \left( 1 + \sum_{m \geq 0} y^{(m)} (-u)^{-m - 1} \right) \left( 1 + \sum_{n \geq 0} y^{(n)} u^{-n - 1} \right)
                    \\
                    & = 1 + \sum_{m \geq 0} y^{(m)} (-u)^{-m - 1} + \sum_{n \geq 0} y^{(n)} u^{-n - 1} + \left( \sum_{m \geq 0} y^{(m)} (-u)^{-m - 1} \right) \left( \sum_{n \geq 0} y^{(n)} u^{-n - 1} \right)
                    \\
                    & = 1 + \sum_{m \geq 0} ( (-1)^{-m - 1} y^{(m)} + y^{(m)} ) u^{-m - 1} + \sum_{m \geq 0} \sum_{n \geq 0} (-1)^{-m - 1} y^{(m)} y^{(n)} u^{-m - n - 2}
                    \\
                    & = 1 + \sum_{l \geq 0} \left( (-1)^{-l - 1} y^{(l)} + y^{(l)} + \sum_{ \substack{m, n \geq 0\\m + n = l} } (-1)^{-m - 1} y^{(m)} y^{(n)} \right) u^{-l - 1}
                    \\
                    & = 1 + \sum_{l \geq 0} \left( 2y^{(2l + 1)} + \sum_{0 \leq m \leq l} (-1)^{-m - 1} y^{(m)} y^{(l - m)} \right) u^{-l - 1}
                \end{aligned}
            $$
        As we are supposed to have:
            $$b(-u) b(u)^{-1} = y(-u) y(u)$$
        we now see - via homogeneity - that the coefficients of $b(u)$ are determined by those of $y(u)$ by:
            \begin{equation} \label{equation: b_cofficients_in_terms_of_y_coefficients}
                2b^{(2l + 1)} - \sum_{ \substack{ 0 \leq m \leq l \\ \text{$m$ even} } } b^{(m)} c^{(l - m)} = 2y^{(2l + 1)} + \sum_{0 \leq m \leq l} (-1)^{-m - 1} y^{(m)} y^{(l - m)}
            \end{equation}
            
        \begin{proposition}[A homomorphism $\UXB(\g_N, \id) \to \calX^{\tw}(\g_N, \id)$] \label{prop: mapping_IMO_reflection_algebras_to_extended_untwisted_yangians_via_extended_twisted_yangians}
            The algebra homomorphism $\Phi$ from proposition \ref{prop: mapping_IMO_reflection_algebras_to_extended_yangians} induces an algebra homomorphism:
                $$\Phi^{\tw}: \UXB(\g_N, \id)[\![u^{-1}]\!] \to \calX^{\tw}(\g_N, \id)[\![u^{-1}]\!]$$
            given by:
                $$\Phi^{\tw}( B(u) ) = S(u)$$
            and fitting into the commutative diagram \eqref{diagram: mapping_IMO_reflection_algebras_to_extended_twisted_yangians_via_untwisted_automorphisms} (and thus also diagram \eqref{diagram: mapping_IMO_reflection_algebras_to_extended_twisted_yangians}).
        \end{proposition}
            \begin{proof}
                This is a direct consequence of lemma \ref{lemma: twisted_quantum_pfaffians}. See the discussion above.
            \end{proof}

        Let us conclude the subsection by considering whether the composition:
            \begin{equation} \label{diagram: mapping_IMO_reflection_algebras_to_BCD0_twisted_yangians}
                \begin{tikzcd}
                    {\UXB(\g_N, \id)[\![u^{-1}]\!]} & {\calX^{\tw}(\g_N, \id)[\![u^{-1}]\!]} & {\calY^{\tw}(\g_N, \id)[\![u^{-1}]\!]}
                    \arrow["{\Phi^{\tw}}", dashed, from=1-1, to=1-2]
                    \arrow[two heads, from=1-2, to=1-3]
                \end{tikzcd}
            \end{equation}
        wherein the unlabelled arrow is the canonical quotient map, is an isomorphism of algebras. 

        Now, in the process of proving \cite[Theorem 4.1]{guay_regelskis_twisted_yangians_for_symmetric_pairs_of_types_BCD}, the authors proved Lemma 4.3, which states that the boundary transfer matrices $S(u) \in \Mat_N( \calX^{\tw}(\g_N, \id)[\![u^{-1}]\!] )$ satisfy the symmetry relation \eqref{equation: symmetry_relation}. This leads us to make the following claim.
        \begin{theorem} \label{theorem: IMO_reflection_algebras_vs_BCD0_twisted_yangians}
            The algebra homomorphism $\Phi^{\tw}$ from proposition \ref{prop: mapping_IMO_reflection_algebras_to_extended_untwisted_yangians_via_extended_twisted_yangians} fits into the following commutative diagram:
                $$
                    \begin{tikzcd}
                    {\UXB(\g_N, \id)[\![u^{-1}]\!]} & {\calX^{\tw}(\g_N, \id)[\![u^{-1}]\!]} \\
                    {\UB(\g_N, \id)[\![u^{-1}]\!]} & {\calY^{\tw}(\g_N, \id)[\![u^{-1}]\!]}
                    \arrow["{{\Phi^{\tw}}}", from=1-1, to=1-2]
                    \arrow[two heads, from=1-1, to=2-1]
                    \arrow[two heads, from=1-2, to=2-2]
                    \arrow["\phi^{\tw}", from=2-1, to=2-2]
                    \end{tikzcd}
                $$
            wherein:
            \begin{itemize}
                \item the left vertical arrow is the canonical quotient map by the ideal generated by the symmetry relation \eqref{equation: symmetry_relation},
                \item the right vertical arrow is the canonical quotient map by the ideal generated by the unitarity relation, and
                \item the bottom arrow $\phi^{\tw}: \UB(\g_N, \id)[\![u^{-1}]\!] \xrightarrow[]{\cong} \calY^{\tw}(\g_N, \id)[\![u^{-1}]\!]$ is the isomorphism from \cite[Theorem 4.1]{guay_regelskis_twisted_yangians_for_symmetric_pairs_of_types_BCD} (see also \cite[Equation 4.32]{guay_regelskis_twisted_yangians_for_symmetric_pairs_of_types_BCD}), which we recall to be given by $\phi^{\tw}( B(u) ) := S(u)$. 
            \end{itemize}
        \end{theorem}
            \begin{proof}
                This is a straightforward diagram chase.
            \end{proof}
        \begin{corollary}[IMO reflection algebras vs. $\BCDzero$ twisted Yangians] \label{coro: IMO_reflection_algebras_vs_BCD0_twisted_yangians}
            The composite homomorphism \eqref{diagram: mapping_IMO_reflection_algebras_to_BCD0_twisted_yangians} is surjective, with kernel equal to the ideal of $\UXB(\g_N, \id)$ generated by the following version of the symmetry relation \eqref{equation: symmetry_relation}:
        \end{corollary}