\section{Quantum (twisted) loop algebras of arbitrary classical affine types}
    \subsection{Trigonometric quantum R-matrices with spectral parameters}
        We begin by recalling some features of quantum R-matrices with spectral parameters, particularly those arising as Baxterisations of trigonometric-type constant quantum R-matrices. 
    
        Again, let:
            $$\g_N \subset \sl_N$$
        be a finite-dimensional simple Lie algebra of a classical type in the Cartan-Killing classification. Let:
            $$\calU_q(\g_N)$$
        be the\footnote{By a Hochschild-cohomological argument this is known to be unique up to isomorphisms as a Hopf algebra deforming the universal enveloping algebra $\calU(\g_N)$.} quantisation of the standard Lie bialgebra structure on $\g_N$. \textit{A priori}, this quantisation is quasi-triangular, thus possessing a (constant) quantum R-matrix:
            $$\bar{\calR}$$
        that is a \textit{constant} solution to the quantum Yang-Baxter equation (QYBE)
            \begin{equation} \label{equation: constant_QYBE}
                \calR_{1, 2} \calR_{1, 3} \calR_{2, 3} = \calR_{2, 3} \calR_{1, 3} \calR_{1, 2}
            \end{equation}
        Equivalently, by the uniqueness of quantisations of Lie bialgebra structures on finite-type Kac-Moody algebras, this is a solution of trigonometric type in the Belavin-Drinfeld classification. This quantum R-matrix is unique to the Hopf algebra $\calU_q(\g_N)$, for it is the Hochschild cohomological class representing this deformation of $\calU(\g_N)$, so we will instead write:
            $$\calU_q(\bar{\calR})$$
        instead of $\calU_q(\g_N)$ in order to put emphasis on the role of the R-matrix. More generally, if $\calr$ is a solution to the classical Yang-Baxter equation (CYBE) \eqref{equation: CYBE} and $\calR \equiv \calr \pmod{\hbar}$ is a solution to the QYBE \eqref{equation: additive_spectral_QYBE} quantising $\calr$, then the corresponding quasi-triangular Hopf algebra obtained via the formalism of Faddeev-Reshetikhin-Takhtajan will be denoted by $\calU_q(\calR)$; this construction is well-known when $\calR$ is a solution to the QYBE with spectral parameter, and for the constant case, see \cite{gautam_rupert_wendlandt_R_matrix_presentation_for_finite_QUEs}).
    
        Next, suppose that $\sfX$ is a connected Dynkin diagram of affine type, of either an untwisted or twisted type, and let:
            $$\calR := \calR(w, z)$$
        be the corresponding universal R-matrix of the quantum Kac-Moody algebra $\calU_q(\sfX)$ (obtained via either the Chevalley-Serre presentation or the Drinfeld current presentation); technically speaking, this depends on the parameter $q$ as well as the Dynkin diagram $\sfX$ (or equivalently, its associated Kac-Moody algebra), but we omit these from the notation to avoid notational cluttering. We remind the reader that $\calR$ is a solution to the spectral QYBE, best written here in its multiplicative form:
            \begin{equation} \label{equation: constant_QYBE}
                \calR_{1, 2}\left(\frac{w}{z}\right) \calR_{1, 3}(w) \calR_{2, 3}(z) = \calR_{2, 3}(z) \calR_{1, 3}(w) \calR_{1, 2}\left(\frac{w}{z}\right)
            \end{equation}
        (one obtains the equation above by making the change of variables $w := \exp(u)$ and $z := \exp(v)$ to \eqref{equation: additive_spectral_QYBE}). Moreover, it is a Baxterisation of the finite-type R-matrix $\bar{\calR}$ from before.

        Now, even though $\calU_q(\calR) := \calU_q(\sfX)$ is unique up to isomorphisms as a quantisation\footnote{In the sense of \cite{etingof_kazhdan_quantisation_1} and its sequels. See \cite{etingof_kazhdan_quantisation_6} in particular.} of the Kac-Moody algebra corresponding to $\sfX$, the QYBE solution $\calR$ thereof is only unique as a cohomological representative. Among other things, this means that there are many different 

    \subsection{Quantum (twisted) loop algebras in the R-matrix presentation}
        Following \cite{guay_regelskis_wendlandt_R_matrix_presentation_of_quantum_loop_algebras}, we work with the following RLL-type construction of quantum (twisted) loop algebras.
        \begin{definition}[Extended quantum loop algebras] \label{def: extended_quantum_loop_algebras}
            The \textbf{extended quantum loop algebra} associated to $\calR$ is the associative algebra:
                $$\calU_q^{\ext}(\Loop^{\sigma}\g_N, \calR)$$
            generated by the coefficients of the matrix entries of the elements:
                $$L^{\pm}(u) \in \Mat_N( \calU_q^{\ext}(\Loop^{\sigma}\g_N, \calR)[\![u^{-1}]\!] )$$
            which are subjected to the following relations:
                \todo[inline]{Upper/lower triangularity}
                $$L_{i, j}^-[0] = L_{j, i}^+[0] = 0, \quad 1 \leq i < j \leq N$$
                $$L_{i, i}^-[0] L_{i, i}^+[0] = L_{i, i}^+[0] L_{i, i}^-[0] = 1, \quad 1 \leq i \leq N$$
                $$
                    \begin{cases}
                        \calR\left(\frac{w}{z}\right) L_1^{\pm}(w) L_2^{\pm}(z) = L_2^{\pm}(z) L_1^{\pm}(w) \calR\left(\frac{w}{z}\right)
                        \\
                        \calR\left(\frac{w}{z}\right) L_1^+(z) L_2^-(w) = L_2^-(z) L_1^+(w) \calR\left(\frac{w}{z}\right)
                    \end{cases}
                $$
        \end{definition}

        \todo[inline]{Hopf structure on $\calU_q^{\ext}(\Loop^{\sigma}\g_N, \calR)$}

        \todo[inline]{Bi-ideal structure on the centre $\calZ_q^{\ext}(\Loop^{\sigma}\g_N, \calR) \subset \calU_q^{\ext}(\Loop^{\sigma}\g_N, \calR)$.}

        \begin{definition}[Quantum loop algebras] \label{def: quantum_loop_algebras}
            The \textbf{quantum loop algebra} associated to $\calR$ is the associative:
                $$\calU_q(\Loop^{\sigma}\g_N, \calR) := \calU_q^{\ext}(\Loop^{\sigma}\g_N, \calR)/\calZ_q(\Loop^{\sigma}\g_N, \calR)$$
        \end{definition}
        \begin{remark}
            $\calU_q(\Loop^{\sigma}\g_N, \calR)$ has a natural Hopf algebra structure inherited from $\calU_q^{\ext}(\Loop^{\sigma}\g_N, \calR)$, and this is because we now know that $\calZ_q(\Loop^{\sigma}\g_N, \calR) \subset \calU_q^{\ext}(\Loop^{\sigma}\g_N, \calR)$ has the structure of a bi-ideal.
        \end{remark}

    \subsection{PBW bases with respect to the R-matrix presentation}