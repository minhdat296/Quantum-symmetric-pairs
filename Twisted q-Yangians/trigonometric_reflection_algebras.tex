\section{Reflection algebras}
    \subsection{Trigonometric K-matrices with spectral parameters}

    \subsection{Reflection equations in quantum (twisted) loop algebras}
        \todo[inline]{Amongst the (generalised) reflection equations, there are two reflection equations of interested, \textbf{the untransposed and transposed ones} (see \cite[Section 6]{regelskis_vlaar_reflection_matrices_coideal_subalgebras}). Molev-Ragoucy-Sorba used the transposed one in their paper \cite{molev_ragoucy_sorba_twisted_q_yangians_type_A} initially, but then commented that the untransposed version would also give rise to the same twisted $q$-Yangians in type $\sfA$. In general, though, these two equations would give rise to different reflection algebras, but it's not clear to me if these reflection algebras are the same up to some kind of "Drinfeld twist". The ambiguity lies within the fact that the (transposed) reflection equations admit many "inequivalent" solutions, i.e. these different K-matrices - unlike the different R-matrices - may give rise to distinct coideal subalgebras of $\calU_q(\Loop^{\sigma}\g_N)$. It is also not clear, if after taking $q \to 1$, one would recover the same usual twisted Yangians. Regardless, \textit{loc. cit.} seems to indicate that it is possible to treat the two cases uniformly.}

        \todo[inline]{Do different Baxterisations of the standard quantum R-matrix of $\rmU_q(\g_N)$ give rise to different families of solutions to the (transposed) reflection equation ?}

    \subsection{PBW bases and embedding into quantum (twisted) loop algebras as associative subalgebras}
        \todo[inline]{My plan is to construct a faithful representation of the reflection algebras from the previous subsection in order to establish PBW bases for these algebras; Ragoucy has preliminary works on vertex representations of reflection algebras that I think can be adapted. Then, I will compare these bases to the PBW bases of quantum loop algebras (which can also be obtained via vertex operators) in order to be able to conclude that the reflection algebras embed as subalgebras into the quantum loop algebras. There is probably a way to do this without having to rely on vertex operators, but it is not clear to me how one would go about doing this.}